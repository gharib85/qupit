% Created 2019-12-20 五 17:32
% Intended LaTeX compiler: pdflatex
\documentclass[11pt]{book}
\usepackage[top=1in,bottom=1in,left=1in,right=1in]{geometry}
\usepackage{graphicx}
\usepackage{amsmath}
\usepackage{amssymb}
\usepackage{bm}
\usepackage{times}
\usepackage{physics}
\usepackage{subfigure}
\usepackage[colorlinks=true,linkcolor=blue,urlcolor=blue,citecolor=blue]{hyperref}
\usepackage{minted}
\setminted{linenos}
\date{}
\title{}
\hypersetup{
 pdfauthor={crf},
 pdftitle={},
 pdfkeywords={},
 pdfsubject={},
 pdfcreator={Emacs 26.3 (Org mode 9.1.9)}, 
 pdflang={English}}
\begin{document}

\title{QUPIT: QUasiadiabatic propagator Path Integral Toolkit}
\author{Ruofan Chen}
\maketitle
\tableofcontents

\chapter{A Review on QUAPI}
\label{sec:orgaa65612}
In this chapter we give a review of the formalism of the QUAPI
\cite{makarov1994-path,makri1995-numerical,segal2010-numerically}.  Let
us consider a generic many-body system which can be modeled by a
finite system of interest coupled to a bath. Let \(H(t)\) denote the
total Hamiltonian which can split into three parts:
\begin{equation}
H(t)=H_S(t)+H_B+H_{SB},
\end{equation}
where \(H_S(t)\) is the Hamiltonian of the system of interest, \(H_B\) is
the Hamiltonian of the bath and \(H_{SB}\) represents the coupling
between the system and the bath. Here we consider the case where only
the system Hamiltonian \(H_S(t)\) is time-dependent. Let \(\rho(t)\) be
the total density matrix, then the time evolution of \(\rho(t)\) is
given by
\begin{equation}
\rho(t)=U(t)\rho(0)U^{\dag}(t),
\end{equation}
where
\begin{equation}
U(t)=\mathrm{T}\exp[-i\int_0^t H(\tau)\dd{\tau}]
=\lim_{\delta t\to0}\prod_{t_i=0}^t e^{-iH(t_i)\delta_t}.
\end{equation}
Here \(\mathrm{T}\) is the chronological ordering symbol, and the
product is understood in that we take the limit over all the
infinitesimal intervals \(\delta t\). Therefore we can write the density
matrix \(\rho(t)\) as
\begin{equation}
\rho(t)=\lim_{N\to\infty}e^{-iH(t_N)\delta t}\cdots e^{-iH(t_0)\delta t}
\rho(0)e^{iH(t_0)\delta t}\cdots e^{iH(t_N)\delta t}
\end{equation}
for \(t_0=0\) and \(\delta t=t/N\).

Now we introduce the reduced density matrix of the system
\(\rho_B(t)=\Tr_B[\rho(t)]\), which is obtained by tracing the total
density matrix over the bath degrees of freedom, then the time
evolution of \(\rho_B(t)\) is given by 
\begin{equation}
\rho_S(s'',s';t)=\Tr_B\bra{s''}
\lim_{N\to\infty}e^{-iH(t_N)\delta t}\cdots e^{-iH(t_1)\delta t}
\rho(0)e^{iH(t_1)\delta t}\cdots e^{iH(t_N)\delta t}\ket{s'}.
\end{equation}
For numerical evaluation we can employ finite \(\delta t\) in the above
expression which approximates the evolution operator \(U(t)\) into a
product of finite \(N\) exponentials. Inserting the identity operator
\(\int\ket{s}\bra{s}\dd{s}\) between every two exponentials and
relabeling \(s'',s'\) as \(s_N^+,s_N^-\) gives
\begin{equation}
\begin{split}
\rho(s_N^+,s_N^-;t)=&\int\dd{s}_0^+\cdots\dd{s_{N-1}^+}\int\dd{s_0^-}\cdots\dd{s_{N-1}^-}\\
&\quad\Tr_B[\mel{s_N^+}{e^{-iH(t_N)\delta t}}{s_{N-1}^+}\cdots
\mel{s_1^+}{e^{-iH(t_1)\delta t}}{s_0^+}\\
&\qquad\qquad\times\mel{s_0^+}{\rho(0)}{s_0^-}
\mel{s_0^-}{e^{iH(t_1)}\delta t}{s_1^-}\cdots
\mel{s_{N-1}^-}{e^{iH(t_N)\delta t}}{s_N^-}].
\end{split}
\end{equation}
The integrand in the above expression can be referred as the
``influence functional'' and we denote it by
\(I(s_0^{\pm},\ldots,s_N^{\pm})\). The influence functional has an
important property which greatly simply the calculation: nonlocal
correlations in the influence functional decay exponentially under
certain conditions. For spin-boson model, finite temperature is
needed, and for spin-fermion model, finite chemical potential
difference or temperature is needed. This property enables practical
numerical calculation for the influence functional.

In the discretized form, \(\rho(0)\) need to ``go'' through \(N\) steps to
arrive at \(\rho(t)\). Then nonlocal correlations with all length need
to be taken into consideration, which means the influence functional
can be written as a product of terms corresponding to different
correlation length \(\Delta k\),
\begin{equation}
\label{eq:IF}
I(s_0^{\pm},\ldots,s_N^{\pm})=\prod_{k=0}^N I_0(s_k^{\pm})
\prod_{k=0}^{N-1}I_1(s_k^{\pm},s_{k+1}^{\pm})\cdots
\prod_{k=0}^{N-\Delta k}I_{\Delta k}(s_k^{\pm},s_{k+\Delta k}^{\pm})\cdots
I_N(s_0^{\pm},s_N^{\pm}).
\end{equation}
Now suppose the influence functional can be truncated beyond a memory
time \(\tau_c=N_s\delta t\) for \(N_s\) a positive integer, then we have
\begin{equation}
\label{eq:IFT}
I(s_0^{\pm},\ldots,s_N^{\pm})=\prod_{k=0}^N I_0(s_k^{\pm})
\prod_{k=0}^{N-1}I_1(s_k^{\pm},s_{k+1}^{\pm})\cdots
\prod_{k=0}^{N-\Delta k}I_{\Delta k}(s_k^{\pm},s_{k+\Delta k}^{\pm})\cdots
\prod_{k=0}^{N-N_s}I_{N_s}(s_0^{\pm},s_{N_s}^{\pm}).
\end{equation}
It is easy to see that \eqref{eq:IFT} becomes \eqref{eq:IF} when
\(N_s\to\infty\), i.e., this approach becomes exact when
\(\tau_c\to\infty\). In addition, we have
\begin{equation}
\begin{split}
I(s_0^{\pm},\ldots,s_N^{\pm})&=I(s_0^{\pm},\ldots,s_{N-1}^{\pm})
I_0(s_N^{\pm})\cdots I_{\Delta k}(s_{N-\Delta k},s_N^{\pm})\cdots
I_{N_s}(s_{N-N_s}^{\pm},s_N^{\pm})\\
&=I(s_0^{\pm},\ldots,s_{N-1}^{\pm})
\frac{I(s_{N-N_s}^{\pm},\ldots,s_{N}^{\pm})}{I(s_{N-N_s}^{\pm},\ldots,s_{N-1}^{\pm})},
\end{split}
\end{equation}
and recursively applying this expression yields
\begin{equation}
\label{eq:IFs}
I(s_0^{\pm},\ldots,s_N^{\pm})=I(s_0^{\pm}\ldots,s_{N_s}^{\pm})
I_s(s_1^{\pm},\ldots,s_{N_s+1}^{\pm})\cdots 
I_s(s_{N-N_s}^{\pm},\ldots,s_N^{\pm}),
\end{equation}
where
\begin{equation}
I_s(s_k^{\pm},\ldots,s_{k+N_s}^{\pm})=
\frac{I(s_k^{\pm},\ldots,s_{k+N_s}^{\pm})}{I(s_k^{\pm},\ldots,s_{k+N_s-1}^{\pm})}.
\end{equation}

To integrate \eqref{eq:IFs}, we define a multiple time system reduced
density matrix \(\tilde{\rho}_S(s_k^{\pm},\ldots,s_{k+N_s-1})\) with an
initial condition \(\tilde{\rho}_S(s_0^{\pm},\ldots,s_{N_s-1}^{\pm})=1\)
for which all initial elements are one. Then the first evolution step
for \eqref{eq:IFs} is given by
\begin{equation}
\tilde{\rho}_S(s_1^{\pm},\ldots,s_{N_s}^{\pm})=
\int I(s_0^{\pm},\ldots,s_{N_s}^{\pm})\dd{s_0^{\pm}},
\end{equation}
and beyond the first step the evolution step is given by 
\begin{equation}
\tilde{\rho}_S(s_{k+1}^{\pm},\ldots,s_{k+N_s}^{\pm})=
\int\tilde{\rho}_S(s_k^{\pm},\ldots,s_{k+N_s-1}^{\pm})
I_s(s_k^{\pm},\ldots,s_{k+N_s}^{\pm})\dd{s_k^{\pm}}.
\end{equation}
Then the time-local \((t_k=k\delta t)\) reduced density matrix is
obtained by summing over all intermediate states for which
\begin{equation}
\rho_S(s_k^+,s_k^-;t_k)=\int\tilde{\rho}_S(s_{k-N_s+1},\ldots,s_k^{\pm})
\dd{s_{k}^{\pm}}\cdots\dd{s_{k-N_s+1}^{\pm}}.
\end{equation}

\section{Spin-Boson Model}
\label{sec:orge256b85}
The system Hamiltonian of the spin-boson model describes a two level
system for which
\begin{equation}
H_S(t)=\frac{1}{2}\bm{B}(t)\cdot\bm{\sigma},
\end{equation}
where \(\bm{\sigma}=(\sigma_x,\sigma_y,\sigma_z)\) which is the Pauli
matrices vector and \(\bm{B}(t)\) is the external field. The bath
Hamiltonian describes a collection of phonon modes for which
\begin{equation}
H_B=\sum_k\omega_kb_k^{\dag}b_k,
\end{equation}
where \(b_k^{\dag}\) (\(b_k\)) creates (annihilates) a phonon with
frequency \(\omega_k\). The system-bath coupling is taken to be
\begin{equation}
H_{SB}=\frac{1}{2}\sum_kV_k(\bm{n}\cdot\bm{\sigma})(b_k+b_k^{\dag}),
\end{equation}
where \(\bm{n}\) is an unit vector with spherical angle \(\theta,\phi\)
for which \(\bm{n}=(\sin\theta\cos\phi,\sin\theta\sin\phi,\cos\theta)\),
and
\begin{equation}
\bm{n}\cdot\bm{\sigma}=\sigma_x\sin\theta\cos\phi+
\sigma_y\sin\theta\sin\phi+\sigma_z\cos\theta.
\end{equation}
This unit vector \(\bm{n}\) describes the system-bath coupling angle
\(\theta,\phi\). The bath can be characterized by the spectral density
\begin{equation}
J(\omega)=\sum_kV_k^2\delta(\omega-\omega_k)=
\alpha\omega^s\Theta(\omega-\omega_c),
\end{equation}
where \(\alpha\) is the control parameter, \(\Theta\) is the heavy step
function and \(\omega_c\) is the cutoff frequency of the phonon modes.
The exponential factor describes the type of bath: when \(0<s<1\) we
have a sub-Ohmic form, when \(s=1\) we have an Ohmic form, and when
\(s>1\) we have a super-Ohmic form.

\chapter{Reference}
\label{sec:orgfc59349}
\bibliographystyle{plainurl}
\bibliography{/home/crf/Nutstore/Literature/papers}
\end{document}